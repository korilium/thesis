%% Generated by Sphinx.
\def\sphinxdocclass{jupyterBook}
\documentclass[letterpaper,10pt,english]{jupyterBook}
\ifdefined\pdfpxdimen
   \let\sphinxpxdimen\pdfpxdimen\else\newdimen\sphinxpxdimen
\fi \sphinxpxdimen=.75bp\relax
%% turn off hyperref patch of \index as sphinx.xdy xindy module takes care of
%% suitable \hyperpage mark-up, working around hyperref-xindy incompatibility
\PassOptionsToPackage{hyperindex=false}{hyperref}
%% memoir class requires extra handling
\makeatletter\@ifclassloaded{memoir}
{\ifdefined\memhyperindexfalse\memhyperindexfalse\fi}{}\makeatother

\PassOptionsToPackage{warn}{textcomp}

\catcode`^^^^00a0\active\protected\def^^^^00a0{\leavevmode\nobreak\ }
\usepackage{cmap}
\usepackage{fontspec}
\defaultfontfeatures[\rmfamily,\sffamily,\ttfamily]{}
\usepackage{amsmath,amssymb,amstext}
\usepackage{polyglossia}
\setmainlanguage{english}



\setmainfont{FreeSerif}[
  Extension      = .otf,
  UprightFont    = *,
  ItalicFont     = *Italic,
  BoldFont       = *Bold,
  BoldItalicFont = *BoldItalic
]
\setsansfont{FreeSans}[
  Extension      = .otf,
  UprightFont    = *,
  ItalicFont     = *Oblique,
  BoldFont       = *Bold,
  BoldItalicFont = *BoldOblique,
]
\setmonofont{FreeMono}[
  Extension      = .otf,
  UprightFont    = *,
  ItalicFont     = *Oblique,
  BoldFont       = *Bold,
  BoldItalicFont = *BoldOblique,
]


\usepackage[Bjarne]{fncychap}
\usepackage[,numfigreset=2,mathnumfig]{sphinx}

\fvset{fontsize=\small}
\usepackage{geometry}


% Include hyperref last.
\usepackage{hyperref}
% Fix anchor placement for figures with captions.
\usepackage{hypcap}% it must be loaded after hyperref.
% Set up styles of URL: it should be placed after hyperref.
\urlstyle{same}


\usepackage{sphinxmessages}



        % Start of preamble defined in sphinx-jupyterbook-latex %
         \usepackage[Latin,Greek]{ucharclasses}
        \usepackage{unicode-math}
        % fixing title of the toc
        \addto\captionsenglish{\renewcommand{\contentsname}{Contents}}
        \hypersetup{
            pdfencoding=auto,
            psdextra
        }
        % End of preamble defined in sphinx-jupyterbook-latex %
        

\title{Future Financial Planning Tools for Consumers}
\date{Jul 31, 2021}
\release{}
\author{Ignace Decocq}
\newcommand{\sphinxlogo}{\vbox{}}
\renewcommand{\releasename}{}
\makeindex
\begin{document}

\pagestyle{empty}
\sphinxmaketitle
\pagestyle{plain}
\sphinxtableofcontents
\pagestyle{normal}
\phantomsection\label{\detokenize{abstract::doc}}


\sphinxAtStartPar
This paper is  written to investigate possible new solution for financial planning tools. Most tools of today lack sufficient theoritical background or are used for commercial purposes. The goal is to try and find better and more comprehensive fields in which financial tools can be better tailored to the consumer. More specifically the recent coöperation between reinforcement learning and planning domains like optimal control will be more closely looked at.


\section{Introduction}
\label{\detokenize{Introduction:introduction}}\label{\detokenize{Introduction::doc}}
\sphinxAtStartPar
The financial decisions that consumers need to make in their present lifetime, become increasingly more complex. A good example of this phenomenon is the shift from defined benefits to defined contributions in which consumers take on greater individual responsibility and risks. The evolution in the abstruseness of financial products has become challenging for consumers who possess low financial knowledge and limiting numeracy skills {[}\hyperlink{cite.Financial_application:id31}{BFH17}{]}. Combined with uncertainty about the future, the consumer is necessitated to be more aware of his financial well\sphinxhyphen{}being than ever before. Looking back into the past, Porteba et al, {[}\hyperlink{cite.Financial_application:id33}{PVW11}{]} conducted an examination of preparedness in retirement for Children of Depression, War Baby, and the Early Baby Boomer in the Health and Retirement Study and Asset and Health Dynamics Among the Oldest Old cohorts. They found that 46.1 percent die with less than 10 000 dollars. With this amount of assets, they would not have the capacity to pay for unexpected events and one might wonder if it is adequate asset levels for retirement. Furthermore, saving behavior has not kept pace with increasing life expectation and the expected prolonged lifespan of the coming generations are unprecedented {[}\hyperlink{cite.Financial_application:id34}{Her11}{]}. All these elements give a painstakingly clear picture that having a vital understanding of one’s financial situation has become one of the greatest challenges in life.

\sphinxAtStartPar
To combat these difficulties, consumers require additional undertakings in planning for their future prosperity. One of the approaches to tackle this issue, is by using financial planning tools. These tools give the consumer the capability to estimate complex intertemporal calculations {[}\hyperlink{cite.Financial_application:id30}{BDTS20}{]}. They also enhance financial behavior, increase household wealth accumulation and they are a complement to other planning aid like a financial advisor {[}\hyperlink{cite.Financial_application:id31}{BFH17}{]}. Although financial planning tools can greatly benefit consumers, it can also be a double\sphinxhyphen{}edged sword. More specifically, when consumers are misinformed about the capabilities of the tool, or when the design of the tool is inadequate, the consumer can be given sub\sphinxhyphen{}optimal advice or even misleading advice {[}\hyperlink{cite.Financial_application:id32}{DMBE18}{]}. Insufficiencies in design can arise when not all essential input variables are included, not all risks are considered, and when accuracy is sacrificed for the ease of use {[}\hyperlink{cite.Financial_application:id30}{BDTS20}{]}. On top of that, there are wide variations in results because of the various methodology and assumptions used in the models {[}\hyperlink{cite.Financial_application:id32}{DMBE18}{]}. For example, assumptions based on inflation and the use of different financial products have a large impact on the results. On the side of the consumer, the possibility of misunderstanding the implications of the results due to a lack of financial knowledge, is a matter of great concern in the eyes of financial educators {[}\hyperlink{cite.Financial_application:id30}{BDTS20}{]}. Clarifying the results is therefore an essential part of making models operational. To improve upon these deficiencies, Dorman et al., {[}\hyperlink{cite.Financial_application:id32}{DMBE18}{]} found that when the models handle additional theoretical variables, the accuracy will improve. Besides, they found that the consumer requires unique solutions that better capture their financial situation. Meaning planning tools need to be more flexible. They should be able to operate in different financial settings and have the ability to look at the impact of changes in input variables. To address the variability in results and the adaptability of models to different settings, this paper will look at reinforcement learning techniques in an intertemporal setting. Reinforcement Learning enables an increase in the flexibility of the model while keeping fundamental theoritical aspects like Optimal Control Theory at its core.

\sphinxAtStartPar
For the remainder of the paper, the general theory of Reinforcement Learning (RL) will first be introduced. Then, some challenges will be discussed together with Deep Reinforcement Learning. Next, The possible implications of RL for financial planning are considered.  Next, a deep Backward Stochastic Differential Equation method is discussed which will solve the Terminal Partial Differential Equation of the dynammic programming system in higher dimensions. Lastly, an example which will implement this method is presented.


\section{Reinforcement Learning}
\label{\detokenize{Reinforcement_learning:reinforcement-learning}}\label{\detokenize{Reinforcement_learning::doc}}
\sphinxAtStartPar
Supervised and unsupervised learning are the two most widely studied and researched branches of Machine Learning (ML). Besides these two, there is also a third subcategorie in ML called Reinforcement Learning (RL). The three branches have fundamental differences between eachother. Supervised learning for example is designed to learn from a training set of labeled data, where each element of the training set describes a certain situation and is linked to a label/action the supervisor has provided {[}\hyperlink{cite.Financial_application:id62}{Ham18}{]}. RL on the other hand is a method in which the machine tries to map situation to actions by maximizing a reward signal {[}\hyperlink{cite.Financial_application:id63}{ADBB17}{]}. The two methods are fundementally different from each other on the fact that in RL there is no supervisor which provides the label/action the machine needs to take, rather there is a reward system set up from which the machine can learn the correct action/label {[}\hyperlink{cite.Financial_application:id62}{Ham18}{]}. contrarily to supervised learning, unsupervised learning tries to find hidden structures within an unlabeled dataset. This might seem similar to RL as both methods work with unlabeled datasets, but RL tries to maximize a reward signal instead of finding only hidden structures in the data {[}\hyperlink{cite.Financial_application:id63}{ADBB17}{]}.

\sphinxAtStartPar
RL finds it roots in multiple research fields. Each of these fields contributes to the RL in its own unique way (see figure) {[}\hyperlink{cite.Financial_application:id62}{Ham18}{]}. For example,  RL is similar to natural learning processes where the method of learning is by experiencing many failures and successes. Therefore psychologists have used RL to mimic psychological processes when an organism makes choices based on experienced rewards/punishments {[}\hyperlink{cite.Financial_application:id69}{EWC21}{]}. While psychologists are mimicing psychological processes, Nueroscientists are using RL to focus on a well\sphinxhyphen{}defined network or regions of the brain that implement value learning {[}\hyperlink{cite.Financial_application:id69}{EWC21}{]}.

\begin{figure}[htbp]
\centering
\capstart

\noindent\sphinxincludegraphics[width=600\sphinxpxdimen,height=250\sphinxpxdimen]{{tree}.png}
\caption{research fields involved in reinforcement learning}\label{\detokenize{Reinforcement_learning:tree-fig}}\end{figure}


\subsection{Finite Markov Decision Processes}
\label{\detokenize{Reinforcement_learning:finite-markov-decision-processes}}
\sphinxAtStartPar
RL can be represented in finite Markov decision processes (MDPs), which are classical formalizations of sequantial decision making. More specifically, MPDs give rise to a structure in which delayed rewards can be balanced with immediate rewards {[}\hyperlink{cite.Financial_application:id70}{SB18}{]}. It also enables a straightforward framing of learning from interaction to achieve a goal {[}\hyperlink{cite.Financial_application:id60}{Lev18}{]}. In it’s most simplest form RL works with an Agent\sphinxhyphen{}Environment Interface. The agent is exposed to some representation of the environment’s state \(S_t \in \mathrm{S}\). From this representation the agent needs to chose an action \( A_t \in \mathcal{A}(s)\), which will result in a numerical reward \(R_{t+1} \in 	\mathbb{R} \) and a new state \(S_{t+1}\) (see figure 2) {[}\hyperlink{cite.Financial_application:id70}{SB18}{]}. The goal for the agent is to learn a mapping from states to action called a policy \(\pi\) that maximizes the expected rewards:
\begin{equation*}
\begin{split} \pi^* = argmax_{\pi} E[R|\pi] \end{split}
\end{equation*}
\sphinxAtStartPar
If the MPDs is finite and discrite, the sets of states, actions and rewards (\(S\), \(A\) , and \(R\)) all have a finite number of elements. The agent\sphinxhyphen{}environment interaction can then be subdivided into episode {[}\hyperlink{cite.Financial_application:id63}{ADBB17}{]}.  The agent’s goal is to maximize the expected discounted cumulative return in the episode {[}\hyperlink{cite.Financial_application:id72}{FranccoisLHI+18}{]}:
\begin{equation*}
\begin{split} G_t = R_{t+1} + \gamma R_{t+2} + \gamma^2 R_{t+3} + ... + \gamma^{T-t-1}R_T = \sum_{k=0}^T \gamma^k R_{t+k+1}\end{split}
\end{equation*}
\sphinxAtStartPar
Where T indicates the terminal state and \(\gamma\) is the discount rate. The terminal state \(S_T\) is mostly followed by a reset to a starting state or sample from a starting distribution of states {[}\hyperlink{cite.Financial_application:id72}{FranccoisLHI+18}{]}. An episode ends once the reset has occured. The discount rate represents the present value of future rewards. If \(\gamma = 0\), the agent is myopic and is only concerned in maximizing the immediate rewards. The agent can consequently be considerd greedy {[}\hyperlink{cite.Financial_application:id70}{SB18}{]}.

\sphinxAtStartPar
The returns can be rewritten in a dynammic programming approach:
\begin{equation*}
\begin{split} G_t = R_{t+1} + \gamma(R_{t+2} + \gamma R_{t+3} + ... + \gamma^{T-t-2}R_T) \end{split}
\end{equation*}\begin{equation*}
\begin{split} G_t = R_{t+1} + \gamma G_{t+1}\end{split}
\end{equation*}
\begin{figure}[htbp]
\centering
\capstart

\noindent\sphinxincludegraphics[width=600\sphinxpxdimen,height=300\sphinxpxdimen]{{standard_model}.png}
\caption{standard model reinforcement learning}\label{\detokenize{Reinforcement_learning:standard-model-fig}}\end{figure}

\sphinxAtStartPar
A key concept of MPDs is the Markov property: Only the current state affects the next state {[}\hyperlink{cite.Financial_application:id72}{FranccoisLHI+18}{]}. The random varianbles (RV) \(R_t\) and \(S_t\) have then well defined discrete probability distributions dependend only on the preceding state and action:
\begin{equation*}
\begin{split} p(s', t| s, a) = Pr(S_t = s', R_t = r | S_{t-1} = s, A_{t-1}=a) \end{split}
\end{equation*}
\sphinxAtStartPar
For all \(s', s \in \mathrm{S} , r \in 	\mathbb{R}, a \in \mathrm{A}(s) \). The probability of each element in the sets \(S\) and \(R\) completely chararcterizes the environment {[}\hyperlink{cite.Financial_application:id70}{SB18}{]}. This can be relaxed by some alogrithms as this is an unrealistic assumption to make. The Partial Observable Markov Decision Process (POMDP) algorithm for example maintains a belief over the current state given the previous belief state, the action taken and the current observation {[}\hyperlink{cite.Financial_application:id63}{ADBB17}{]}.  Once \(p\) is known, the environment is fully discribed and functions like a transition function \(T : D \times A \to p(S)\) and a reward function \(R: S \times A \times S \to \mathbb{R}\) can be deducted {[}\hyperlink{cite.Financial_application:id70}{SB18}{]}.

\sphinxAtStartPar
Most algorithms in RL use a value function to estimate the value of a given state for the agent. Value functions are defined by the policy \(\pi\) the agent has decided to take. As mentioned previously, \(\pi\) is the mapping of states to probabilities of selecting an action. The value function \(v_{\pi}(s)\) in a state \(s\) following a policy \(\pi\) is as followes:
\begin{equation*}
\begin{split} v_{\pi}(s) = E_{\pi}[G_t | S_t = s] = E_{\pi}[\sum_{k=0}^T \gamma^kR_{t+k+1} | S_t=s] \end{split}
\end{equation*}
\sphinxAtStartPar
This can aso be rewritten in a dynammic programming approach:
\begin{equation}\label{equation:Reinforcement_learning:my_label}
\begin{split}v_{\pi}(s) = E_{\pi}[G_t | S_t = s] \\
= E_{\pi}[R_{t+1} + \gamma G_{t+1} | S_t = s] \\
= \sum_a \pi(a|s) \sum_{s'} \sum_r p(s', r|s,a)[r + \gamma E_{pi}[G_{t+1} | S_{t+1} = s'] \\
= \sum_a \pi(a|s) \sum_{s', r}p(s', r|s,a)[r + \gamma v_{\pi}(s')| S_{t+1} = s'] \end{split}
\end{equation}
\sphinxAtStartPar
The formula is called the Bellman equation of \(v_{\pi}\). It describes the relationschip between the value of a state and the values of its successor states given a certain policy \(\pi\). The relation can also be represented by a backup diagram (see figure 3). If \(v_{\pi}(s)\) is the value of a given state, then \(q_{\pi}(s,a)\) is the value of a given action of that state:
\begin{equation*}
\begin{split} q_{\pi}(s,a) = E_{\pi}[G_t | S_t = s, A_t = a] = E_{\pi}[\sum_{k=0}^T \gamma^kR_{t+k+1} | S_t=s, A_t = a] \end{split}
\end{equation*}
\sphinxAtStartPar
This can be seen in the backup diagram as starting from the black dot and cumputing the subsequential value thereafter. \(q_{\pi}(s,a)\) is also called the action\sphinxhyphen{}value function as it describes each value of an action for each state.

\begin{figure}[htbp]
\centering
\capstart

\noindent\sphinxincludegraphics[width=300\sphinxpxdimen,height=250\sphinxpxdimen]{{backup_diagram}.png}
\caption{General backup diagram}\label{\detokenize{Reinforcement_learning:backup-diagram-fig}}\end{figure}

\sphinxAtStartPar
For the agent it is important to find the optimal policy in which it maximizes the expected cumulative rewards. The optimal policy \(\pi_*\) is the policy for which \(v_{\pi_*}(s) > v_{\pi}(s)\) for all \(s \in S\). An optimal policy also has the same action\sphinxhyphen{}value function \(q_*(s,a)\) for all \(s \in S\) and \(a \in A\). The optimal policy does not depend soley on one policy and can encompass multiple policy. It is thus not policy dependend:
\begin{equation*}
\begin{split} v_*(s) = max_{a \in A(s)} q_{\pi_*}(s,a) \end{split}
\end{equation*}\begin{equation*}
\begin{split} = max_{a} E_{\pi_*}[G_t | S_t=s, A_t=a] \end{split}
\end{equation*}\begin{equation*}
\begin{split} = max_{a} E_{\pi_*}[R_{t+1} + \gamma G_{t+1} | S_t=s, A_t=a] \end{split}
\end{equation*}\begin{equation*}
\begin{split} = max_{a} E[R_{t+1} + \gamma v_*(S_{t+1}) | S_t=s, A_t=a] \end{split}
\end{equation*}
\sphinxAtStartPar
Once \(v_*(s)\) is found, you just need to apply a greedy algorithm as the optimal value function already takes into account the long\sphinxhyphen{}term consequences of choosing that action. Finding \(q_*(s,a)\) makes things even easier, as the action\sphinxhyphen{}value function caches the result of all one\sphinxhyphen{}step\sphinxhyphen{}ahead searches.

\sphinxAtStartPar
Solving the Bellman equation of the value function or the action\sphinxhyphen{}value function such that we know each all possibilities with their probabilities and rewards is in most practical cases not possible. Typical due to three main factors {[}\hyperlink{cite.Financial_application:id70}{SB18}{]}. The first problem is obtaining full knowledge of the dynamics of the environment. The second factor is the computational resources to complete the calculation. the last factor is that the states need to have the markov property.   To circumvent these obstacles RL tries to approximate the Bellman optimality equation using various methods. In the next chapter, a brief layout of theser method is discussed with a focus on the methods applicable for financial planning.


\subsection{model\sphinxhyphen{}based RL, model\sphinxhyphen{}free RL and planning}
\label{\detokenize{Reinforcement_learning:model-based-rl-model-free-rl-and-planning}}
\sphinxAtStartPar
A general theory in finding the optimal policy \(\pi_*\) is called Generelized Policy Iteration (GLI). This method is applied to almost all RL algorithms. The main idea behind GLI is that there is a process which evaluates the value function of the current policy \(\pi\) called policy evaluation and a process which improves the current value function called policy improvement {[}\hyperlink{cite.Financial_application:id70}{SB18}{]}. To find the optimal policy these two processes work in tandem with eachother as seen in figure … Counterintuitively, these processes also work in a conflicting manner as policy improvement makes the policy incorrect and it is thus no longer the same policy {[}\hyperlink{cite.Financial_application:id70}{SB18}{]}. While policy evaluations creates a consistent policy and thus the policy no longer improves upon itself. This idea runs in parallel with the balance between exploration and exploitation in RL.  If the focus lies more on exploration, the agent frequently tries to find states which improve the value function. However, putting more emphasis on exploration is a costly setting as the agent will more frequenlty choose suboptimal policies to explore the state space. If exploitation is prioritised, the agent will take a long time to find the optimal policy as the agent is likely not to explore new states to improve the policy.  is a good example of the influential balance between exploration and exploitation.

\begin{figure}[htbp]
\centering
\capstart

\noindent\sphinxincludegraphics[width=500\sphinxpxdimen,height=300\sphinxpxdimen]{{GPI}.png}
\caption{Generalized policy iteration}\label{\detokenize{Reinforcement_learning:gpi-fig}}\end{figure}

\sphinxAtStartPar
Reinforcement Learning can be subdivided between model\sphinxhyphen{}based RL and model\sphinxhyphen{}free RL. In model\sphinxhyphen{}free RL the dynamics of the environment are not known. \(\pi_*\) is found by purily interacting with the environment. Meaning that these algorithms do not use transition probability distribution and reward function related to MDP. Moreover, model\sphinxhyphen{}free RL have irreversible access to the environment. Meaning the algorithm has to move forward after an action is taken. Model\sphinxhyphen{}based RL on the other hand have reversible access to the environment because they are able to revert the model and make another trail from the same state {[}\hyperlink{cite.Financial_application:id24}{MBJ20}{]}. Good examples of model\sphinxhyphen{}free RL techniques are the Q\sphinxhyphen{}learning and Policy Optimization algorithms. They tend to be used on a variety of tasks, like playing video games to learning complicated locomotion skills. Model\sphinxhyphen{}free RL lay at the fundation of RL and are one of the first algorithms to be applied in RL. On the other hand, model\sphinxhyphen{}based RL is developed independently and in parallel with planning methods like optimal control and the search community as they both solve the same problem but differ in the approach. Most algorithms in model\sphinxhyphen{}based RL have a model which describes the dynamics of the environment. They sample from that model to then improve a learned value or policy function {[}\hyperlink{cite.Financial_application:id24}{MBJ20}{]}(see figure). This enables the agent to think in advance and as it were plan for possible actions. Model\sphinxhyphen{}based reinforcement learning finds thus large similarities with the Planning literature and as a result a lot of cross breeding between the two is happening. For example an extension of the POMP algorithm called Partially Observable Multi\sphinxhyphen{}Heuristic Dynammic Programming (POMHDP) is based on recent progress from the search community {[}\hyperlink{cite.Financial_application:id27}{KSL19}{]}. A hybrid version of the two approaches in which the model is learned through interaction with the environment, has also been widely applied. The imagination\sphinxhyphen{}augmented agents (12A) for example combines model\sphinxhyphen{}based and model\sphinxhyphen{}free aspects by employing the predictions as additional context in a deep policy network.  In the next subsection three fundamental algorithms in RL are discussed which will enable us to better capture the dimensions and challenges of a RL algorithms.

\begin{sphinxVerbatim}[commandchars=\\\{\}]

```\PYGZob{}figure\PYGZcb{} C:/Users/ignac/Documents/GitHub/thesis/notebook/images/model\PYGZus{}based\PYGZus{}reinforcement\PYGZus{}learning.png
\PYGZhy{}\PYGZhy{}\PYGZhy{}
height: 350px
width: 500px
name: MB\PYGZus{}RL\PYGZhy{}fig
\PYGZhy{}\PYGZhy{}\PYGZhy{}
model\PYGZus{}based reinforcement learning  
```
\end{sphinxVerbatim}

\begin{sphinxVerbatim}[commandchars=\\\{\}]
  File \PYGZdq{}\PYGZlt{}ipython\PYGZhy{}input\PYGZhy{}1\PYGZhy{}4133c88b0d52\PYGZgt{}\PYGZdq{}, line 1
    ```\PYGZob{}figure\PYGZcb{} C:/Users/ignac/Documents/GitHub/thesis/notebook/images/model\PYGZus{}based\PYGZus{}reinforcement\PYGZus{}learning.png
    \PYGZca{}
SyntaxError: invalid syntax
\end{sphinxVerbatim}


\subsubsection{Dynammic Programming, Monte Carlo Methods and Temporal\sphinxhyphen{}Difference Learning}
\label{\detokenize{Reinforcement_learning:dynammic-programming-monte-carlo-methods-and-temporal-difference-learning}}
\sphinxAtStartPar
Dynammic Programming (DP) is known for two algorithms in RL: value iteration (VI) and policy iteration (PI). For both methodes the dynamics of the environment need to be completly known and they therefore fall under model\sphinxhyphen{}based RL. The two algorithms also use a discrete time, state and action MDP as they are iterative procedures. The PI can be subdivided into three steps: initialize, policy evaluation and policy improvement. The first step is to initialize the value function \(v_{\pi}\) by choosing an arbitrary policy \(\pi\). The following step is to evaluate the function successively by updating the the Bellman equation eq 2.1 . Updating on the Bellman equation is also called the expected update as the equation is updated using the whole state space instead of a sample of the state space. One update is also called a sweep as the update sweeps through the state space. Now that we have updated the value function \(v_{\pi}\), we know how good it is to follow the current policy. The next step is to deviate from the policy trajectory and chose a different action a in state s to find a more optimal policy value. We compute the new \(\pi '\) and compare it to the old policy. The new policy is accepted if \(\pi '(s) > \pi(s)\). This process is repeated untill a convergence criteria is met. The complete algorithm can be found in the appendix. VI combines the policy evaluation with the policy improvement by truncating the sweep with one update of each state. It effectivily combines the policy evaluation and policy evaluation in one sweep (see appendix for algorithm) {[}\hyperlink{cite.Financial_application:id70}{SB18}{]}. PI and VI are the foundation of DP and numerous adaptions have been made on these algorithms. For example have… . Adaptive Dynammic programming is

\sphinxAtStartPar
The Monte Carlo (MC) methods do not assume full knowledge of the dynamics of the environment and are thus considered model\sphinxhyphen{}free RL techniques. They only require a sample sequence of states, actions and rewards from interaction of an environment. Techniquely, a model is still required which generates sample transitions, but the complete probability distribtion \(p\) of the dynammic system is not neccesary. The idea behind almost all MC methods is that the agent learns the optimal policy by averaging the sample returns of a policy \(\pi\). They can therefore not learn on an online basis as after each episode they need to average their returns. Another difference between the two methods is that the MC method does not bootstrap like DP. Meaning, each state has an independed estimate. Note that Monte Carlo methods create a nonstationary problem as each action taken at a state depends on the previous states. MC methods can either estimate  a state value (eq) or  estimate the value of a state\sphinxhyphen{}action pairs (eq) (recall that the state\sphinxhyphen{}action values are the value of an action given a state). If state values are estimated, a model is required as it needs to be able to look ahead one step and choose the action which leads to the best reward and next state. With action value estimation you already estimated the value of the action and no model needs to be taken into account.  Monte Carlo methods also use a term called visits. A visit is when a state or state\sphinxhyphen{}action pair is in the sample path. Multiple visits to a state are possible in an episode. Two general Monte Carlo Methods can be deducted from visits. The every\sphinxhyphen{}visit MC methods and the first\sphinxhyphen{}visit MC methods. The every\sphinxhyphen{}visit MC methods estimates the value of a state as the average of the returns that have followed all visits to it. The first visit method only looks at the first visit of that state to estimate the average returns. The biggest hurdle in MC methods is that most state\sphinxhyphen{}action pairs might never be visited in the sample.

\sphinxAtStartPar
To overcome this problem multiple solutions have been explored. The naïve solution to this problem is called the exploring starts. Here, the idea is to allocate to each action in each state a nonzero probability at the start of the process. Although this is not possible in a practical setting where we truly want to interact with an environment, it enables us to improve to policy by making it greedy with respect to the current value function. If an infinite number of episodes are taken, the policy improvement theory states that the policy \(\pi\) will convergence too the optimal policy \(\pi_*\) given the exploring starts. The other two possibilities are on\sphinxhyphen{}policy methods and off\sphinxhyphen{}policy methods {[}\hyperlink{cite.Financial_application:id70}{SB18}{]}. On\sphinxhyphen{}policy methods attempt to improve on the current policy. This is also called a soft policy as \(\pi(a|s) > 0\) for all \(s \in S\) and all \( a \in A(s)\), but shifts eventual to the deterministic optimal policy. One of these on\sphinxhyphen{}policy methods uses a \(\varepsilon\)\sphinxhyphen{}greedy policy. The \(\varepsilon\)\sphinxhyphen{}greedy policy uses with probability \(\varepsilon\) a random action instead of the greedy action. A pseudocode of on\sphinxhyphen{}policy first visit MC for \(\varepsilon\)\sphinxhyphen{}soft policies algorithm can be found in the appendix.


\subsection{Challenges in RL and deep RL}
\label{\detokenize{Reinforcement_learning:challenges-in-rl-and-deep-rl}}
\sphinxAtStartPar
{[}\hyperlink{cite.Financial_application:id24}{MBJ20}{]} adresses the six most important dimensions of a RL algortihm: computational effort, action value selection, cumulative return estimation, policy evaluation, function representation and update method. The first dimension has to do with the computational effort that is required to run the algorithm. This has primarely to do with the state set that is chosen (see figure). The first option is to consider all states \(S\) of the dynamic environment. In practice this often becomes impractical to consider due to the curse of dimensionality. The second and third possibilities are all reachable states and all relevant states. All reachable states are the states which are reachable from any start under any policy, while for the relevant states only those state under the optimal policy are considered. The last option is to use start states. These are all the states with a non\sphinxhyphen{}zero probability under \(p(s_0)\)

\sphinxAtStartPar
(need examples and further explanaition curse of dimensionality)

\begin{figure}[htbp]
\centering
\capstart

\noindent\sphinxincludegraphics[width=500\sphinxpxdimen,height=350\sphinxpxdimen]{{state_space}.png}
\caption{state\_space dimensions}\label{\detokenize{Reinforcement_learning:state-space-fig}}\end{figure}

\sphinxAtStartPar
The second dimension is the action selection and has primarly to due to with exploration process of the algorithm. The first consideration is the candidate set that is considered for the next action. Then the optimal action needs to be considered while still keeping exploration in mind. For selecting the candidate set two main approaches are considered: step\sphinxhyphen{}wise and frontier. Frontier methods only start exploration once they are on the frontier, while step\sphinxhyphen{}wise methods have a new candidate set at each step of the trajectory. the second consideration, selecting the acion value,  different methods have been adopted. The first one are random explorations like \(\varepsilon\)\sphinxhyphen{}greedy exploration as explained in the section of Monte Carlo methods. These explorations techniques enable us to escape from a local minimum but can cause a jittering effect in which we undo an exploration step at random. The second approach is value\sphinxhyphen{}based exploration which uses the value\sphinxhyphen{}based information to better direct the pertubation. A good example of this are mean action values. They improve the random exploration by incorporating the mean estimates of all the available actions. Meaning, they explore actions with higher values more frequenlty than actions with lower values. The last option is state\sphinxhyphen{}based exploration. State\sphinxhyphen{}based exploration uses state\sphinxhyphen{}dependedent properties to inject noise. The dynammic programming (section…) is a good example of this approach. DP is an ordered state\sphinxhyphen{}based exploration. Ordered state\sphinxhyphen{}based exploration sweeps through the state space in orded like tree structure. Other state\sphinxhyphen{}based exploration are possible like novelty and priors.

\sphinxAtStartPar
The calculation of the cumulative return estimation (eq) can be reformulated to adress the practical issues and limitations in RL:
\begin{equation*}
\begin{split} G_t = \sum_{k=0}^T \gamma^kR_{t+k+1} + \gamma^KB(s_{t+T}) \end{split}
\end{equation*}
\sphinxAtStartPar
Where \(T \in {1,2,3, ..., \infty}\) denotes the sample depth and \(B(.)\) is a bootstrap function. For the sample depth three possible option are possible: \(K = \infty\), \(k = 1\),  \(k = n\) or reweighted. Monte Carlo methods for example use a sample depth to infinity as they do not bootstrap at all. Instead, DP uses bootstrapping at each iteration, so \(k = 1\). An intermediate method between DP and Monte Carlo methods can also be deviced in which \( k = n\). The reweighted option is a special case of \( k = n\) in which tragets of different depths are combined with a weighting scheme. The bootstrap function can be devised using a learned value function like state value function or the state\sphinxhyphen{}action value function or following a heuristic approach. A good heuristic can be obtained by first solving a simplified version of the problem. An example of this is first solvinfDthe deterministic problem and then using the solution as a heirstic on its stochastic counterpart.

\sphinxAtStartPar
The fourth dimension to consider is policy evaluation.


\subsection{Reinforcement learning and financial planning}
\label{\detokenize{Reinforcement_learning:reinforcement-learning-and-financial-planning}}

\section{Financial Application of Reinforcement Learning}
\label{\detokenize{Financial_application:financial-application-of-reinforcement-learning}}\label{\detokenize{Financial_application::doc}}

\subsection{Optimal consumption, investment and life insurance in an intertemporal model}
\label{\detokenize{Financial_application:optimal-consumption-investment-and-life-insurance-in-an-intertemporal-model}}
\sphinxAtStartPar
The first person to include uncertain lifetime and life insurance decisions in a discrete life\sphinxhyphen{}cycle model was Yaari {[}\hyperlink{cite.Financial_application:id19}{Yaa65}{]}. He explored the model using a utility function without bequest (Fisher Utility function) and a utility function with bequest (Marshall Utility function) in a bounded lifetime. In both cases, he looked at the implications of including life insurance. Although Yaari’s model was revolutionary in the sense that now the uncertainty of life could be modeled, Leung {[}\hyperlink{cite.Financial_application:id23}{Leu94}{]} found that the constraints laid upon the Fisher utility function were not adequate and lead to terminal wealth depletion. Richard {[}\hyperlink{cite.Financial_application:id22}{Ric75}{]} applied the methodology of Merton {[}\hyperlink{cite.Financial_application:id20}{Mer69}, \hyperlink{cite.Financial_application:id21}{Mer75}{]} to the problem setting of Yaari in a continuous time frame. Unfortunately, Richard’s model had one deficiency: The bounded lifetime is incompatible with the dynamic programming approach used in Merton’s model. As an individual approaches his maximal possible lifetime T, he will be inclined to buy an infinite amount of life insurance. To circumvent this Richard used an artificial condition on the terminal value. But due to the recursive nature of dynamic programming, modifying the last value would imply modifying the whole result. Ye {[}\hyperlink{cite.Financial_application:id25}{Ye06}{]}  found a solution to the problem by abandoning the bounded random lifetime and replacing it with a random variable taking values in \([0,\infty)\). The models that replaced the bounded lifetime, are thereafter called intertemporal models as the models did not consider the whole lifetime of an individual but rather looked at the planning horizon of the consumer.  Note that the general setting of Ye {[}\hyperlink{cite.Financial_application:id25}{Ye06}{]} has a wide range of theoretical variables, while still upholding a flexible approach to different financial settings. On this account, it is a good baseline to confront the issues concerning the current models of financial planning. However, one of the downsides of the model is the abstract representation of the consumer. Namely, the rational consumer is studied, instead of the actual consumer. To detach the model from the notion of rational consumer, I will more closely look at behavioral concepts that can be implemented. In the next paragraph various modification will be discussed and a further review is conducted on the behavioral modifications

\sphinxAtStartPar
After Ye {[}\hyperlink{cite.Financial_application:id25}{Ye06}{]} various models have been proposed which all have given rise to unique solutions to the consumption, investment, and insurance problem. The first unique setting is a model with multiple agents involved. For example,  Bruhn and Steffensen {[}\hyperlink{cite.Financial_application:id35}{BS11}{]} analyzed the optimization problem for couples with correlated lifetimes with their partner nominated as their beneficiary using a copula and common\sphinxhyphen{}shock model, while Wei et al.{[}\hyperlink{cite.Financial_application:id38}{WCJW20}{]} studied optimization strategies for a household with economically and probabilistically dependent persons. Another setting is where certain constraints are used to better describe the financial situation of consumers. Namely, Kronborg and Steffensen {[}\hyperlink{cite.Financial_application:id37}{KS15}{]} discussed two constraints. One constraint is a capital constraint on the savings in which savings cannot drop below zero. The other constrain involves a minimum return in savings. A third setting describes models who analyze the financial market and insurance market in a pragmatic environment. A good illustration is the study of Shen and Wei {[}\hyperlink{cite.Financial_application:id39}{SW16}{]}. They incorporate all stochastic processes involved in the investment and insurance market where all randomness is described by a Brownian motion filtration. An interesting body of models is involved in time\sphinxhyphen{}inconsistent preferences. In this framework, consumers do not have a time\sphinxhyphen{}consistent rate of preference as assumed in the economic literature. There exists rather a divergence between earlier intentions and later choices De\sphinxhyphen{}Paz et al. {[}\hyperlink{cite.Financial_application:id42}{DPMSNR14}{]}. This concept is predominantly described in psychology. Specifically, rewards presented closer to the present are discounted proportionally less than rewards further into the future. An application of time\sphinxhyphen{}inconsistent preferences in the consumption, investment, and insurance optimization can be found in Chen and Li {[}\hyperlink{cite.Financial_application:id41}{CL20}{]} and De\sphinxhyphen{}Paz et al. {[}\hyperlink{cite.Financial_application:id42}{DPMSNR14}{]}. These time\sphinxhyphen{}inconsistent preferences are rooted in a much deeper behavioral concept called future self\sphinxhyphen{}continuity. Future self\sphinxhyphen{}continuity can be described as how someone sees himself in the future. In classical economic theory, we assume that the degree to which you identify with yourself has no impact on the ultimate result. In the next subsection, the relationship of future self\sphinxhyphen{}continuity and time\sphinxhyphen{}inconsistent preferences are more closely looked at and future self\sphinxhyphen{}continuity is further examined in the behavioral life\sphinxhyphen{}cycle model.


\subsubsection{The model specifications}
\label{\detokenize{Financial_application:the-model-specifications}}
\sphinxAtStartPar
In this section, I will set the dynamics for the baseline model in place. The dynamics follow primarily from the paper of Ye {[}\hyperlink{cite.Financial_application:id25}{Ye06}{]}.

\sphinxAtStartPar
Let the state of the economy be represented by a standard Brownian motion \(w(t)\), the state of the consumer’s wealth be characterized by a finite state multi\sphinxhyphen{}dimensional continuous\sphinxhyphen{}time Markov chain \(X(t)\) and let the time of death be defined by a non\sphinxhyphen{}negative random variable \(\tau\). All are defined on a given probability space (\(\Omega, \mathcal{F}, P\)) and \(W(t)\) is independent of \(\tau\). Let \(T< \infty\) be a fixed planning horizon. This can be seen as the end of the working life for the consumer. \(\mathbb{F} = \{\mathcal{F}_t, t \in [0,T]\}\), be the P\sphinxhyphen{}augmentation of the filtration \(\sigma\)\{\(W(s), s<t \}, \forall t \in [0,T]\) , so \(\mathcal{F}_t\) represents the information at time t. The economy consist of a financial market and an insurance market. In the following section I will construct these markets separetly.

\sphinxAtStartPar
The financial market consist of a risk\sphinxhyphen{}free security \(B(t)\) and a risky security \(S(t)\), who evolve according to
\begin{equation*}
\begin{split} \frac{dB(t)}{B(t)}=r(t)dt \end{split}
\end{equation*}\begin{equation*}
\begin{split} \frac{dS(t)}{S(t)}=\mu(t)dt+\sigma(t)dW(t)\end{split}
\end{equation*}
\sphinxAtStartPar
Where \(\mu, \sigma, r > 0\) are constants and \(\mu(t), r(t), \sigma(t): [0,T] \to R\) are continous. With \(\sigma(t)\) satisfying \(\sigma^2(t) \ge k, \forall t \in [0,T]\)

\sphinxAtStartPar
The random variable \(\tau_d\) needs to be first modeled for the insurance  market. Assume that \(\tau\) has a probability density function \(f(t)\) and probability distribution function given by
\begin{equation*}
\begin{split} F(t) \triangleq P(\tau < t) = \int_0^t f(u) du \end{split}
\end{equation*}
\sphinxAtStartPar
Assuming \(\tau\) is independent of the filtration \(\mathbb{F}\)

\sphinxAtStartPar
Following on the probability distribution function we can define the survival function as followed
\begin{equation*}
\begin{split} \bar{F}(t)\triangleq P(\tau \ge t) = 1 -F(t) \end{split}
\end{equation*}
\sphinxAtStartPar
The hazard function is the  instantaneous death rate for the consumer at time t and is defined by
\begin{equation*}
\begin{split} \lambda(t) = \lim_{\Delta t\to 0} = \frac{P(t\le\tau < t+\Delta t| \tau \ge t)}{\Delta t} \end{split}
\end{equation*}
\sphinxAtStartPar
where \(\lambda(t): [0,\infty[ \to R^+\) is a continuous, deterministic function with \(\int_0^\infty \lambda(t) dt = \infty\).

\sphinxAtStartPar
Subsequently, the survival and probability density function can be characterized by
\begin{equation*}
\begin{split} \bar{F}(t)= {}_tp_0= e^{-\int_0^t \lambda(u)du} \end{split}
\end{equation*}\begin{equation*}
\begin{split} f(t)=\lambda(t) e^{-\int_0^t\lambda(u)du} \end{split}
\end{equation*}
\sphinxAtStartPar
With conditional probability described as
\begin{equation*}
\begin{split} f(s,t) \triangleq \frac{f(s)}{\bar{F}(t)}=\lambda(s) e^{-\int_t^s\lambda(u)dy} \end{split}
\end{equation*}\begin{equation*}
\begin{split} \bar{F}(s,t) = {}_sp_t \triangleq \frac{\bar{F}(s)}{\bar{F}(t)} = e^{-\int_t^s \lambda(u)du} \end{split}
\end{equation*}
\sphinxAtStartPar
Now that \(\tau\) has been modeled, the life insurance market can be constructed. Let’s assume that the life insurance is continuously offered and that it provides coverage for an infinitesimally small period of time. In return, the consumer pays a premium rate p when he enters into a life insurance contract, so that he might insure his future income. In compensation he will receive  a total benefit of \(\frac{p}{\eta(t)}\) when he dies at time t. Where \(\eta : [0,T] \to R^+ \) is a continuous, deterministic function.

\sphinxAtStartPar
Both markets are now described and the wealth process \(X(t)\) of the consumer can now be constructed. Given an initial wealth \(x_0\), the consumer receives a certain amount of income \(i(t)\) \(\forall t \in [0,\tau \wedge T]\) and satisfying \(\int_0^{\tau \wedge T} i(u)du < \infty\). He needs to choose at time t a certain premium rate \(p(t)\), a certain consumption rate \(c(t)\) and a certain amount of his wealth \(\theta (t)\) that he invest into the risky asset \(S(t)\). So given the processes \(\theta\), c, p and i, there is a wealth process \(X(t)\)  \(\forall t \in [0, \tau \wedge T] \) determined by
\begin{equation*}
\begin{split} dX(t) = r(t)X(t) + \theta(t)[( \mu(t) - r(t))dt +\sigma(t)dW(t)] -c(t)dt -p(t)dt + i(t)dt,   \quad t \in [0,\tau \wedge T] \end{split}
\end{equation*}
\sphinxAtStartPar
If \(t=\tau\) then the consumer will receive the insured amount \(\frac{p(t)}{\eta(t)}\). Given is wealth X(t) at time t his total legacy will be
\begin{equation*}
\begin{split} Z(t) = X(t) + \frac{p(t)}{\eta(t)} \end{split}
\end{equation*}
\sphinxAtStartPar
The predicament for the consumer is that he needs to chose the optimal rates for c, p , \(\theta\) from the set \(\mathcal{A}\) , called the set of admissible strategies, defined by
\begin{equation*}
\begin{split} \mathcal{A}(x) \triangleq  \textrm{set of all possible triplets (c,p,}\theta) \end{split}
\end{equation*}
\sphinxAtStartPar
such that his expected utility from consumption, from legacy when \(\tau > T\) and from terminal wealth when \(\tau \leq T \)  is maximized.
\begin{equation*}
\begin{split} V(x) \triangleq \sup_{(c,p,\theta) \in \mathcal{A}(x)} E\left[\int_0^{T \wedge \tau} U(c(s),s)ds + B(Z(\tau),\tau)1_{\{\tau \ge T\}} + L(X(T))1_{\{\tau>T\}}\right] \end{split}
\end{equation*}
\sphinxAtStartPar
Where \(U(c,t)\) is the utility function of consumption, \(B(Z,t)\) is the utility function of legacy and \(L(X)\) is the utility function for the terminal wealth. \(V(x)\) is called the value function and the consumers wants to maximize his value function by choosing the optimal set \(\mathcal{A} = (c,p,\theta)\). The optimal set \(\mathcal{A}\) is found by using the dynamic programming technique described in the following section.


\subsubsection{dynamic programming principle}
\label{\detokenize{Financial_application:dynamic-programming-principle}}
\sphinxAtStartPar
To solve the consumer’s problem the value function needs to be restated in a dynamic programming form.
\begin{equation*}
\begin{split}J(t, x; c, p, \theta) \triangleq E \left[\int_0^{T \wedge \tau} U(c(s),s)ds + B(Z(\tau),\tau)1_{\{\tau \ge T\}} + L(X(T))1_{\{\tau>T\}}| \tau> t, \mathcal{F}_t \right] \end{split}
\end{equation*}
\sphinxAtStartPar
The value function becomes
\begin{equation*}
\begin{split} V(t,x) \triangleq \sup_{\{c,p,\theta\} \in \mathcal{A}(t,x)} J(t, x; c, p, \theta)  \end{split}
\end{equation*}
\sphinxAtStartPar
Because \(\tau\) is independent of the filtration, the value function can be rewritten as
\begin{equation*}
\begin{split} E \left[\int_0^T  \bar{F}(s,t)U(c(s),s) + f(s,t)B(Z(\tau),\tau) ds  + \bar{F}(T,t)L(X(T))| \mathcal{F}_t \right]\end{split}
\end{equation*}
\sphinxAtStartPar
The optimization problem is now converted from a random  closing time point to a fixed closing time point. The mortality rate can also be seen as a discounting function for the consumer as he would value the utility on the probability of survival.

\sphinxAtStartPar
Following the dynamic programming principle we can rewrite this equation as the value function at time s plus the value created from time step t to time step s. This enables us to view the optimization problem into a time step setting, giving us the incremental value gained at each point in time.
\begin{equation*}
\begin{split} V(t,x) = \sup_{\{c,p,\theta\} \in \mathcal{A}(t,x)} E\left[e^{-\int_t^s\lambda(v)dv}V(s,X(s)) + \int_t^s f(s,t)B(Z(s),s) + \bar{F}(s,t)U(c(s),s)ds|\mathcal{F}_t\right] \end{split}
\end{equation*}
\sphinxAtStartPar
The Hamiltonian\sphinxhyphen{}Jacobi\sphinxhyphen{}bellman (HJB) equation can be derived from the dynamic programming principle and is as follows
\begin{gather*}
\begin{cases} 
V_t(t,x) -\lambda V(t,x) + \sup_{(c,p,\theta)} \Psi(t,x;c,p,\theta)  = 0 \\ V(T,x) = L(x)  
\end{cases}
\end{gather*}
\sphinxAtStartPar
where
\begin{equation*}
\begin{split} \Psi(t,x; c,p,\theta) \triangleq r(t)x + \theta(\mu(t) -r(t)) + i(t) -c -p)V_x(t,x) + \\ \frac{1}{2}\sigma^2(t)\theta^2V_{xx}(t,x) + \lambda(t)B(x+ p/\eta(t),t) + U(c,t) \end{split}
\end{equation*}
\sphinxAtStartPar
Proofs for deriving the HJB equation, dynammic programming principle and converting from a random closing time point to a fixed closing time point can be found in Ye {[}\hyperlink{cite.Financial_application:id25}{Ye06}{]}

\sphinxAtStartPar
A strategy is optimal if
\begin{gather*}
0 =V_t(t,x) -\lambda(t)V(t,x) + \sup_{c,p,\theta}(t,x;c,p,\theta)  \\
0 = V_t(t,x) -\lambda(t)V(t,x) + (r(t)x+ i(t))V_x + \sup_c\{U(c,t)-cV_x\} + \\ \sup_p\{\lambda(t)B(x + p/\eta(t),t) - pV_x\} + \sup_\theta \{ \frac{1}{2}\sigma^2(t)V_{xx}(t,x)\theta^2 +(\mu(t) - r(t))V_x(t,x)\theta\} 
\end{gather*}
\sphinxAtStartPar
The first order conditions for regular interior maximum are
\begin{equation*}
\begin{split}\sup_c  \{ U(c,t) - cV_x\} = \Psi_c(t,x;c^*,p^*,\theta^*)  \rightarrow  0 = -V_x(t,x) + U_c(c*,t) \end{split}
\end{equation*}\begin{equation*}
\begin{split} \sup_p\{\lambda(t)B(x + p/\eta(t),t) - pV_x\} = \Psi_p(t,x;c^*,p^*,\theta^*) \\ \rightarrow 0 = -V_x(t,x) + \frac{\lambda(t)}{\eta{t}}B_Z(x + p^*/\eta(t),t) \end{split}
\end{equation*}\begin{equation*}
\begin{split} \sup_\theta \{ \frac{1}{2}\sigma^2(t)V_{xx}(t,x)\theta^2 +(\mu(t) - r(t))V_x(t,x)\theta\} = \Psi_\theta(t,x;c^*,p^*,\theta^*)\\ \rightarrow 0 = (\mu(t) -r(t))V_x(t,x) + \sigma^2(t)\theta^*V_{xx}(t,x) \end{split}
\end{equation*}
\sphinxAtStartPar
The second order conditions are
\begin{equation*}
\begin{split} \Psi_{cc}, \Psi_{pp}, \Psi_{\theta \theta} < 0 \end{split}
\end{equation*}

\subsubsection{The analytic result: Constant Relative Risk Aversion utility function}
\label{\detokenize{Financial_application:the-analytic-result-constant-relative-risk-aversion-utility-function}}
\sphinxAtStartPar
This optimal control problem has been solved analytically by Ye {[}\hyperlink{cite.Financial_application:id25}{Ye06}{]} for the Constant Relative Risk Aversion utility function. In this paper however, I will use the analytical result derived by Ye to compare the performance of the NeuralNetDiffEq algorithm and see whether this convergences to the analytical result derived by Ye. Once this is established, other utility function might be used for solving the optimal control problem using the NeuralNetDiffEq algorithm.


\subsection{Algorithm BFSDE}
\label{\detokenize{Financial_application:algorithm-bfsde}}
\sphinxAtStartPar



\section{discussion}
\label{\detokenize{Discussion:discussion}}\label{\detokenize{Discussion::doc}}
\sphinxAtStartPar
the


\section{Appendix}
\label{\detokenize{Appendix:appendix}}\label{\detokenize{Appendix::doc}}

\subsection{pseudocode algorithms}
\label{\detokenize{Appendix:pseudocode-algorithms}}
\sphinxAtStartPar
\sphinxincludegraphics{{policy_iteration}.png}

\sphinxAtStartPar
\sphinxincludegraphics{{value_iteration}.png}

\begin{sphinxthebibliography}{Franccoi}
\bibitem[ADBB17]{Financial_application:id63}
\sphinxAtStartPar
Kai Arulkumaran, Marc Peter Deisenroth, Miles Brundage, and Anil Anthony Bharath. A brief survey of deep reinforcement learning. \sphinxstyleemphasis{arXiv preprint arXiv:1708.05866}, 2017.
\bibitem[BFH17]{Financial_application:id31}
\sphinxAtStartPar
Qianwen Bi, Michael Finke, and Sandra J Huston. Financial software use and retirement savings. \sphinxstyleemphasis{Journal of Financial Counseling and Planning}, 28(1):107–128, 2017.
\bibitem[BDTS20]{Financial_application:id30}
\sphinxAtStartPar
Rachel Qianwen Bi, Lukas R Dean, Jingpeng Tang, and Hyrum L Smith. Limitations of retirement planning software: examining variance between inputs and outputs. \sphinxstyleemphasis{Journal of Financial Service Professionals}, 2020.
\bibitem[BS11]{Financial_application:id35}
\sphinxAtStartPar
Kenneth Bruhn and Mogens Steffensen. Household consumption, investment and life insurance. \sphinxstyleemphasis{Insurance: Mathematics and Economics}, 48(3):315–325, 2011.
\bibitem[CL20]{Financial_application:id41}
\sphinxAtStartPar
Shou Chen and Guangbing Li. Time\sphinxhyphen{}inconsistent preferences, consumption, investment and life insurance decisions. \sphinxstyleemphasis{Applied Economics Letters}, 27(5):392–399, 2020.
\bibitem[DPMSNR14]{Financial_application:id42}
\sphinxAtStartPar
Albert De\sphinxhyphen{}Paz, Jesus Marin\sphinxhyphen{}Solano, Jorge Navas, and Oriol Roch. Consumption, investment and life insurance strategies with heterogeneous discounting. \sphinxstyleemphasis{Insurance: Mathematics and Economics}, 54:66–75, 2014.
\bibitem[DMBE18]{Financial_application:id32}
\sphinxAtStartPar
Taft Dorman, Barry S Mulholland, Qianwen Bi, and Harold Evensky. The efficacy of publicly\sphinxhyphen{}available retirement planning tools. \sphinxstyleemphasis{Available at SSRN 2732927}, 2018.
\bibitem[EWC21]{Financial_application:id69}
\sphinxAtStartPar
Maria K Eckstein, Linda Wilbrecht, and Anne GE Collins. What do reinforcement learning models measure? interpreting model parameters in cognition and neuroscience. \sphinxstyleemphasis{Current Opinion in Behavioral Sciences}, 41:128–137, 2021.
\bibitem[FranccoisLHI+18]{Financial_application:id72}
\sphinxAtStartPar
Vincent François\sphinxhyphen{}Lavet, Peter Henderson, Riashat Islam, Marc G Bellemare, and Joelle Pineau. An introduction to deep reinforcement learning. \sphinxstyleemphasis{arXiv preprint arXiv:1811.12560}, 2018.
\bibitem[Ham18]{Financial_application:id62}
\sphinxAtStartPar
Ahmad Hammoudeh. A concise introduction to reinforcement learning. 2018.
\bibitem[Her11]{Financial_application:id34}
\sphinxAtStartPar
Hal E Hershfield. Future self\sphinxhyphen{}continuity: how conceptions of the future self transform intertemporal choice. \sphinxstyleemphasis{Annals of the New York Academy of Sciences}, 1235:30, 2011.
\bibitem[KSL19]{Financial_application:id27}
\sphinxAtStartPar
Sung\sphinxhyphen{}Kyun Kim, Oren Salzman, and Maxim Likhachev. Pomhdp: search\sphinxhyphen{}based belief space planning using multiple heuristics. In \sphinxstyleemphasis{Proceedings of the International Conference on Automated Planning and Scheduling}, volume 29, 734–744. 2019.
\bibitem[KS15]{Financial_application:id37}
\sphinxAtStartPar
Morten Tolver Kronborg and Mogens Steffensen. Optimal consumption, investment and life insurance with surrender option guarantee. \sphinxstyleemphasis{Scandinavian Actuarial Journal}, 2015(1):59–87, 2015.
\bibitem[Leu94]{Financial_application:id23}
\sphinxAtStartPar
Siu Fai Leung. Uncertain lifetime, the theory of the consumer, and the life cycle hypothesis. 1994.
\bibitem[Lev18]{Financial_application:id60}
\sphinxAtStartPar
Sergey Levine. Reinforcement learning and control as probabilistic inference: tutorial and review. \sphinxstyleemphasis{arXiv preprint arXiv:1805.00909}, 2018.
\bibitem[Mer69]{Financial_application:id20}
\sphinxAtStartPar
Robert C Merton. Lifetime portfolio selection under uncertainty: the continuous\sphinxhyphen{}time case. \sphinxstyleemphasis{The review of Economics and Statistics}, pages 247–257, 1969.
\bibitem[Mer75]{Financial_application:id21}
\sphinxAtStartPar
Robert C Merton. Optimum consumption and portfolio rules in a continuous\sphinxhyphen{}time model. In \sphinxstyleemphasis{Stochastic Optimization Models in Finance}, pages 621–661. Elsevier, 1975.
\bibitem[MBJ20]{Financial_application:id24}
\sphinxAtStartPar
Thomas M Moerland, Joost Broekens, and Catholijn M Jonker. A framework for reinforcement learning and planning. \sphinxstyleemphasis{arXiv preprint arXiv:2006.15009}, 2020.
\bibitem[PVW11]{Financial_application:id33}
\sphinxAtStartPar
James M Poterba, Steven F Venti, and David A Wise. Were they prepared for retirement? financial status at advanced ages in the hrs and ahead cohorts. In \sphinxstyleemphasis{Investigations in the Economics of Aging}, pages 21–69. University of Chicago Press, 2011.
\bibitem[Ric75]{Financial_application:id22}
\sphinxAtStartPar
Scott F Richard. Optimal consumption, portfolio and life insurance rules for an uncertain lived individual in a continuous time model. \sphinxstyleemphasis{Journal of Financial Economics}, 2(2):187–203, 1975.
\bibitem[SW16]{Financial_application:id39}
\sphinxAtStartPar
Yang Shen and Jiaqin Wei. Optimal investment\sphinxhyphen{}consumption\sphinxhyphen{}insurance with random parameters. \sphinxstyleemphasis{Scandinavian Actuarial Journal}, 2016(1):37–62, 2016.
\bibitem[SB18]{Financial_application:id70}
\sphinxAtStartPar
Richard S Sutton and Andrew G Barto. \sphinxstyleemphasis{Reinforcement learning: An introduction}. MIT press, 2018.
\bibitem[WCJW20]{Financial_application:id38}
\sphinxAtStartPar
Jiaqin Wei, Xiang Cheng, Zhuo Jin, and Hao Wang. Optimal consumption–investment and life\sphinxhyphen{}insurance purchase strategy for couples with correlated lifetimes. \sphinxstyleemphasis{Insurance: Mathematics and Economics}, 91:244–256, 2020.
\bibitem[Yaa65]{Financial_application:id19}
\sphinxAtStartPar
Menahem E Yaari. Uncertain lifetime, life insurance, and the theory of the consumer. \sphinxstyleemphasis{The Review of Economic Studies}, 32(2):137–150, 1965.
\bibitem[Ye06]{Financial_application:id25}
\sphinxAtStartPar
Jinchun Ye. \sphinxstyleemphasis{Optimal life insurance purchase, consumption and portfolio under an uncertain life}. University of Illinois at Chicago, 2006.
\end{sphinxthebibliography}







\renewcommand{\indexname}{Index}
\printindex
\end{document}